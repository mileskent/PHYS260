\documentclass[10pt, a4paper, twocolumn]{article}

\input{structure.tex}

% TITLE PAGE
\title{PHYS 260 Electromagnetism}
\author{\authorstyle{Miles Kent}}
\date{}

\begin{document}
\maketitle

    \section{Electrostatics Basics}    
        \subsection{Introduction}
            \begin{enumerate}
                \item {
                    As you may know, matter is made of atoms.
                    Atoms have a nucleus, which is made up of protons (positive) and neutrons (neutral), and electrons (negative).
                    Opposite charges are attracted to one another with a inverse square force, and like charges are similarly repelled.
                }
                \item {
                    The electrons\footnote{Not \textit{all} of the electrons in a conductor can move, it's just that the ones that don't move can be ignored because they do not contribute to the net charge of a material} in conductors are able to move freely, while in insulators they cannot unless they come off entirely
                }
                \item {
                    Some materials tend to give off electrons (e.g. fur) and some tend to attract them (e.g. plastic)   
                }
                \item {
                    Not all metals are conductors
                }
                \item {
                    A neutral object can be attracted to a charged object via inducted polarity. In a conductor, this means the electrons move through the entire object to create a dipole. In an insulator, the electrons of the atoms realign to create billions of of tiny dipoles. This makes a difference due to the inverse square law, which will be discussed further later. The former creates a strong attraction, while the latter creates a weak attraction.
                }
            \end{enumerate}
        \subsection{Electrical Grounding}
            An object can be "grounded" by connecting it to a big conductor, which serves as a resevoir for charge, e.g. you can ground the wall outlet by sticking a fork into it. The gruond in this case is you body and the ground. The electricity from the wall will flow through your body and into the ground


    \section{Coulomb's Law}    
        The magnitude of the force between two point charges $q_1$ and $q_2$ C, with distance $r$ m is equal to the following
        \begin{align*} 
            F = \frac{1}{4\pi \epsilon_0} \frac{|q_1 \cdot q_2|}{r^2}
        \end{align*} 
        \begin{itemize}
            \item where $\epsilon_0$ is vacuum permittivity, the value of the absolute dielectric permittivity of classical vacuum, or also just "the electric constant".  
            \item $\epsilon_0 \approx 8.854189 \cdot 10^{-12}$ 
            \item the Coulomb constant $k = \frac{1}{4\pi \epsilon_0}$
            \item electric forces add as vectors, which is called the "Superposition Principle"
        \end{itemize}


    \section{Electric Fields}
        \begin{itemize}
            \item Electric Fields are best represented by vector fields

            \item For the electric field of a point charge\\ $\vec{F} = q \vec{E} \rightarrow E = \frac{F}{q} \rightarrow E = \frac{1}{4\pi \epsilon_0} \frac{q}{r^2} = \frac{kq}{r^2}\ \frac{N}{C}$

            \item Field lines do not indicate the trajectory of a test charge, but rather are lines tangent to the electric field at a given point
        \end{itemize}

        \subsection{Electric Field Integrations}    
            As it turns out, it is unrealistic to use Coulomb's Law just by itself to calculate the electric field at a point. This would require knowing the location and charge of every charge in an object. One way this problem can be simplified is with the use of charge density and calculus. 
            \begin{itemize}
                \item Linear density: $\lambda = \frac{Q}{L}$
                \item Area density: $\sigma = \frac{Q}{A}$
                \item Volumetric density: $\rho = \frac{Q}{V}$
            \end{itemize}

        \subsubsection{Lines of Charge}    
        Finite line
        Infinite line
        \subsubsection{Sheets of Charge}    
        Finite sheet (circle)
        Infinite sheet

    \section{Dipoles}    

    \section{Gauss' Law}

\end{document}
